\chapter{Desafios} \label{c:desafios}

Dada a larga escala de impacto do sistema proposto por este trabalho, seria impossível esperar que nenhum desafio estivesse entre a situação atual da internet e das ferramentas disponíveis e o estágio desejado e vislumbrado pelos projetos de grandes cidades inteligentes. Na verdade, são vários os desafios em cada uma das camadas de implementação sugeridas, sejam eles técnicos, financeiros, sociais, ou culturais.

\section{Privacidade}

Assim como introduzido na seção \ref{ss:privacidade_na_internet_das_coisas}, considerando a imensa dependência da solução proposta e de projetos de cidades inteligentes de uma excelente e abrangente camada de coleta de dados, privacidade há de ser um dos maiores desafios socioculturais impostos à web semântica e internet das coisas \cite{Kirrane2018PrivacySA}.

Seria impossível atingir excelência de coleta de dados se, por exemplo, todos os canais de dados expostos na seção \ref{s:fontes_de_dados_consideradas} fossem considerados violações de direitos universais de privacidade e propriedade de informação, e portanto proibidos ou banidos de sistemas de cidades inteligentes.

Espera-se que um processo de conscientização, esclarecimento e regulamentação detalhado aconteça nos próximos anos \cite{security:web3.0} - assim como o que teve início com o GDPR (\textit{General Data Protection Regulation}) na Europa - a fim de permitir que a população tenha total ciência dos riscos \textbf{e benefícios} da disponibilização de acesso de dados privados a serviços de terceiros.

\section{Escalabilidade}

Sistemas de recomendação de conteúdo clássicos fazem parte de um privilegiado grupo de técnicas utilizadas neste trabalho que não sofre de problemas de escalabilidade, pois possuem um escopo de \textbf{coleta, armazenamento e processamento} de dados bem definido e limitado.

A inclusão do conceito de metadados - aqui incluindo as dimensões temporais e geoespaciais - eleva a escala destas três etapas ao mundo de sistemas distribuídos multiagentes de processamento em tempo real. Daí, considerar a análise multidimensional para a definição de contexto e para o enriquecimento deste já alcança um patamar de volume de dados para processamento interessante, onde as ferramentas atuais se veem desafiadas. 

Por último, a inclusão deste projeto num cenário de cidades inteligentes ultrapassa as barreiras de volume, variedade e velocidade de dados processáveis com tranquilidade em múltiplas ordens de grandeza\footnote{\href{https://www.zdnet.com/article/volume-velocity-and-variety-understanding-the-three-vs-of-big-data/}{ZDNet: Understanding the three V's of big data}}.

A maior conclusão no tópico de escalabilidade e factibilidade técnica é a de que será necessário um nível de multidisciplinaridade altíssimo, além do uso sincronizado de múltiplas ferramentas especialistas, sempre num modelo de sistemas altamente distribuídos e não-concorrentes.

Qualquer outra tentativa de procura ou utilização de uma ferramenta que resolva todos os problemas com excelência pode colocar o processo de tomada de decisão de sistemas conectados em perigo\footnote{\url{https://www.forbes.com/sites/brentdykes/2017/06/28/big-data-forget-volume-and-variety-focus-on-velocity}}.

Aplicações na área da saúde, assim como na área de petróleo, também demonstram que o problema de escalabilidade pode vir por outra interpretação: dados excessivos. O centro de pesquisa e consultoria McKinsey concluiu, em 2015\footnote{\href{https://www.mckinsey.com/business-functions/digital-mckinsey/our-insights/the-internet-of-things-the-value-of-digitizing-the-physical-world}{McKinsey - Unlocking the potential of the Internet of Things}}, que em alguns casos apenas 1\% da informação gerada por grandes ecossistemas de IoT, com milhares de dispositivos de coleta, é processado e transformado em valor real para aplicações. Por isso, é importante que a estrutura de governança algorítmica e a infraestrutura de uma cidade inteligente estejam alinhadas para ter volume, granularidade, e valor dos dados controlados e bem definidos, a fim de não sobrecarregar as etapas de processamento e enriquecimento de dados\footnote{\url{https://www.iotforall.com/iot-data-visualization-tools/}}\footnote{\url{https://www.iotforall.com/iot-healthcare-data-context/}}.

Por fim, o trabalho feito anteriormente na disciplina de Smart Grid e Smart Cities no apêndice \ref{a:BigDataSmartCity} reforça como e onde os problemas de escalabilidade podem surgir, assim como discorre sobre a necessidade de um ecossistema multidisciplinar e multiagente, tendo várias tecnologias diferentes, cada uma especialista em uma abordagem, a fim de atingir o nível de escalabilidade e eficiência necessário para o desenvolvimento de um sistema completo para uma cidade inteligente.

\section{Efeitos do mundo digital na evolução humana}

A teoria da seleção natural e processos evolutivos tem sido colocada em cheque na era digital, onde diversas habilidades do ser humano que o diferenciou competitivamente ao longo de centenas de milhares de anos estão se perdendo pela crescente dependência do homem para com as máquinas e sistemas digitais de informação.

Ferramentas de assistência cognitiva, como o motor de busca Google, possuem a maior parcela nessa influência\footnote{\url{https://neurocritic.blogspot.com/2011/07/google-stroop-effect.html}} quando o problema é observado com foco nas últimas duas décadas \cite{Sparrow2011GoogleEO}. 

Vários livros foram escritos sobre a influência da tecnologia sobre os processos biológicos, psicológicos e especialmente cognitivos \cite{theshallows, theglasscage}. Não existe unanimidade: alguns autores detalham uma visão negativa da relação do homem com o processo evolutivo\footnote{\url{http://www.ucl.ac.uk/media/library/humanevolution}}; outros descrevem de maneira otimista como o ser humano tem tomado as rédias do processo evolutivo e está decidindo seu próprio destino evolutivo\footnote{\url{http://www.kurzweilai.net/evolution-and-the-internet-toward-a-networked-humanity}}\footnote{\href{https://www.nationalgeographic.com/magazine/2017/04/evolution-genetics-medicine-brain-technology-cyborg/}{National Geographic: How Humans Are Shaping Our Own Evolution}}.