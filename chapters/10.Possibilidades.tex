\chapter{Possibilidades e trabalhos futuros} \label{c:possibilidades_trabalhos_futuros}

Tendo descrito as peças necessárias para compor um sistema completo de coleta, processamento, enriquecimento e aplicação de dados no cenário de cidades inteligentes, podemos mencionar algumas iniciativas e projetos que chamam a atenção pelo potencial de amadurecer o sistema aqui proposto de maneira relevante, e talvez até de atacar e resolver vários dos desafios apresentados ao decorrer do projeto.

\section{SOLID Pods}

Tim Berners-Lee, o pesquisador que arquitetou em 1989, enquanto membro da Organização Europeia para a Pesquisa Nuclear (\textit{Conseil Européen pour la Recherche Nucléaire}, CERN), a primeira versão do que hoje chamamos de \textbf{Web} (a parte mais visível da internet), vem descrevendo modelos da web semântica há mais de 20 anos, com os primeiros rascunhos, discussões online, e livros \cite{berners2001weaving} datando de 1998\footnote{\url{https://www.w3.org/DesignIssues/Semantic.html}}.

A partir do seu trabalho de pesquisa pelo MIT, sendo membro fundador do World Wide Web Consortium (W3C), Tim vem desenvolvendo o modelo de um novo conjunto de ferramentas e abordagens que definitivamente se encaixam em parte do que a suposta Web 3.0 tem como premissa. Em novembro de 2015, o Computer Science and Artificial Intelligence Lab (CSAIL) do MIT publicou\footnote{\href{https://www.csail.mit.edu/news/web-inventor-tim-berners-lees-next-project-platform-gives-users-control-their-data}{CSAIL, MIT, News - Web inventor Tim Berners-Lee's next project}} que o projeto SOLID havia ganho um presente de 1 milhão de dólares da MasterCard como incentivo de pesquisa para o projeto e para o Grupo de Informação Descentralizada (Decentralized Information Group, DIG\footnote{\url{http://dig.csail.mit.edu/}}).

Segue a definição do projeto pelo site do mesmo, em tradução livre do inglês:

\begin{quote}
    Solid (derived from "social linked data") is a proposed set of conventions and tools for building decentralized social applications based on Linked Data principles. Solid is modular and extensible and it relies as much as possible on existing W3C standards and protocols.
\end{quote}

\begin{quote}
    Solid (derivado de "dados sociais interligados") é um conjunto de convenções e ferramentas proposto para construir aplicações sociais descentralizadas baseadas nos princípios de Dados Interligados. Solid é modular e extensível, e se baseia o máximo possível em padrões e protocolos existentes da W3C.
\end{quote}e

O projeto SOLID foi anunciado desde 2016 várias vezes\footnote{\url{https://www.digitaltrends.com/web/ways-to-decentralize-the-web/}}\footnote{\url{https://www.wired.com/2017/04/tim-berners-lee-inventor-web-plots-radical-overhaul-creation/}} pela mídia como uma possível "solução para a web"\footnote{\href{https://www.extremetech.com/extreme/281334-tim-berners-lees-solid-project-can-it-save-the-web}{ExtremeTech - Tim Bernes-Lee's Solid Project: Can it save the web?}}.

Dentre as origens e influências do projeto, estão um projeto de armazenamento social nas nuvens\footnote{\url{https://www.w3.org/DesignIssues/CloudStorage.html}} e um modelo de dados relacionados de escrita e leitura\footnote{\url{https://www.w3.org/DesignIssues/ReadWriteLinkedData.html}}, ambos pela W3C. A primeira obra de pesquisa sobre o projeto é de 2016, com coautoria do próprio Tim Berners-Lee \cite{mansour2016demonstration}. Alguns trabalhos derivados já existem, fazendo uso da plataforma e projeto SOLID para definição de um modelo de notificações com dados interconectados \cite{capadisli2017linked}.

Uma explicação sobre o modo de funcionamento e convenções da ferramenta pode ser encontrado no portal da Inrupt\footnote{\url{https://solid.inrupt.com/how-it-works}}, companhia do próprio Tim Berners-Lee que oferece serviços na plataforma.

\section{Sidewalk Labs}

Filha do conjunto de empresas que compõem a Alphabet Inc. (assim como a Google Inc.), a empresa norte americada Sidewalk Labs é, por definição, uma \textbf{organização de inovação urbana}. Seu objetivo é reimaginar cidades para melhorar a qualidade de vida de seus moradores\footnote{\url{https://www.sidewalklabs.com/}}. Sendo uma das empresas irmãs da Google, e sabendo que esta é uma das maiores presenças no mundo de coleta, processamento, enriquecimento e aplicação de dados no universo da web semântica, a Sidewalk Labs e seus projetos de cidades inteligentes nativamente conectadas têm o potencial de amadurecer vários modelos intermediários ou completos de um sistema como o proposto por este trabalho. Seu projeto principal e inicial é o de revitalização da área de Quayside em Toronto\footnote{\url{https://sidewalktoronto.ca/}}, no Canadá, já nos moldes que definem uma cidade inteligente.

Saber que os projetos da Sidewalk Labs, e futuramente outros projetos de cidades inteligentes derivados, estão nascendo com acesso a todo o conhecimento, \textit{expertise} e tecnologia disponível e amadurecido pela Google, dá uma credibilidade enorme à capacidade desses projetos de alcançarem níveis de arquitetura e eficiência de implementação das etapas descritas neste trabalho.